\documentclass[12pt]{article}
\usepackage[top=2cm, bottom=3cm, left=3cm, right=3cm]{geometry}
\usepackage[utf8]{inputenc}
\usepackage{charter}
\usepackage[T1]{fontenc}
\usepackage[french]{babel}
\usepackage{hyperref}
\usepackage{parskip}
\usepackage{comment}

\begin{comment}
Informations sur le formatage du document

Section et sous-sections:
Chaque section est définie par un \section{} et chaque sous-section par un \subsection{}.
Toute déclaration d'une section ou d'une sous-section doit être suivie d'un \label{} pour pouvoir être référencée.
Le format d'un label est le suivant: sec:<nom de la section> pour une section et sec:<nom de la section>:<nom de la sous-section> pour une sous-section.
Les sections sont référencées à l'aide de \autoref{<nom de la référence>}.
\end{comment}


% Informations sur l'association
\newcommand{\asso}{Nom de l'asso}
\newcommand{\dateCreation}{1 juin 2022}

\title{Règlement intérieur \asso}
\author{}
\date{\today}

\begin{document}
\maketitle

\vspace*{10cm}

Dernière modification: \today.

Ce Règlement Intérieur a pour objectif de préciser les Statuts de l’association \asso, ci-après dénommée
\asso, dont l’objet est de \textbf{XXXX}. Ce Règlement Intérieur
devra être connu par chaque nouvel adhérent.

\section{Membres}
\label{sec:membres}

\subsection{Composition}
\label{sec:membres:composition}

L’association \asso~est composée des membres suivants :
\begin{itemize}
    \item Membres du Bureau
    \item Membres du Conseil d’Administration
    \item Membres d’honneur
    \item Membres ponctuels
    \item Membres bienfaiteurs
    \item Membres adhérents
\end{itemize}

\subsection{Cotisation}
\label{sec:membres:cotisation}

Les membres d’honneur ne paient pas de cotisation.
Le montant de celle-ci est fixé par le Conseil d’Administration selon la procédure suivante :
\begin{itemize}
	\item la cotisation sera définie par année universitaire de TELECOM Nancy et sera redevable par chaque
    cotisant au début de chaque année
	\item le montant de la cotisation est proposé par le Président ou 2 (deux) des membres du Conseil
d’Administration
	\item le vote de la nouvelle cotisation se fait à la majorité des deux-tiers
	\item la nouvelle cotisation prend effet à partir du lendemain du vote en Conseil d’Administration, et la
modification sera reportée sur le bulletin d’adhésion afin que les nouveaux adhérents puissent en être
informés
\end{itemize}

Le montant de la cotisation est actuellement fixé à 0€ (zéro euros) par ce Règlement Intérieur.
Toute cotisation versée à l’association est définitivement acquise. Aucun remboursement de cotisation ne
peut être exigé en cas de démission, d’exclusion ou de décès d’un membre en cours d’année.

\subsection{Admission de membres nouveaux}
\label{sec:membres:admission}

L’association peut à tout moment accueillir de nouveaux membres. L’adhésion à l’association entraîne
l’acceptation des clauses prévues aux Statuts et au présent Règlement Intérieur. La procédure d’admission
dépend du type de membre considéré :

\subsubsection{Membres du Conseil d’Administration}
\label{sec:membres:admission:conseil}

En cas de vacance, les nouveaux administrateurs de l’association seront élus par le Conseil d’Administration
parmi les candidats étudiants de TELECOM Nancy.
L’admission ne sera effective qu’après la signature du bulletin d’adhésion.

\subsubsection{Membres d’honneur}
\label{sec:membres:admission:honneur}




Ces membres sont désignés par le Conseil d’Administration par un vote à la majorité simple après proposition
du candidat par un administrateur.
Un membre d’honneur est dispensé de cotisation ainsi que de bulletin d’adhésion. Il n’est pas forcément
étudiant à TELECOM Nancy.
La qualité de membre d’honneur ne se perd pas en cas de réintégration au sein de l’association.

\subsection{Droit des membres}
\label{sec:membres:droit}

\subsubsection{Droit à l’image}
\label{sec:membres:droit:image}

Lors de son adhésion, l’étudiant autorise ou non \asso~à le prendre en photo et en vidéo, et à utiliser ces
contenus à des fins de communication ou dans le cadre de ses activités.
Conformément à la loi du 18 mars 1803, l’étudiant peut demander la non-publication ou la suppression de
toute photographie sur laquelle il est identifiable.

\subsubsection{Protection des données à caractère personnel}
\label{sec:membres:droit:données}

Les données fournies par les étudiants lors de leur inscription font l’objet de traitements informatiques. En
application de la loi 78-17 du 6 janvier 1978 modifiée et du règlement européen n° 2016/679, dit Règlement
Général sur la Protection des Données (RGPD), l’étudiant bénéficie de droits quant aux données à caractère
personnel pouvant être collectées :
\begin{itemize}
	\item Droit d’accès : Demande d’information sur les données détenues par \asso
	\item Droit de rectification : rectification des informations détenues par \asso
	\item Droit de suppression : effacement des données détenues par \asso
	\item Droit d’opposition : opposition à la conservation, transmission ou diffusion des données détenues par
\asso
	\item Droit de portabilité : récupération des données détenues par \asso~et transmission à une autre
entité
\end{itemize}

Pour exercer ces droits, il convient d’adresser une demande par mail au Secrétaire Général de \asso
joignable à l’adresse suivante : humanitn@telecomnancy.eu.
Concernant un membre de \asso, les données collectées sont :
\begin{itemize}
	\item son nom
	\item son prénom
	\item son adresse email
	\item ses date et lieu de naissance
	\item une photocopie de sa carte étudiante pour l’année en cours
	\item son adresse postale
	\item son numéro de téléphone
\end{itemize}

Ces données sont collectées dans le but de communiquer au sein de l’association et de remplir des déclaratifs
pouvant être fait à l’URSSAF. Elles pourront également être utilisées lorsque l’étudiant ne sera plus membre
de la structure afin d’alimenter un réseau des anciens de \asso.

\subsubsection{Fin de mandat}
\label{sec:membres:droit:fin_mandat}


Lors d’une Assemblée Générale, le Président entrant peut déléguer son pouvoir à un autre membre du bureau
entrant ou sortant afin de permettre une simplification des actions en cours. Cette délégation de pouvoir ne
pourra être effective qu’un mois au maximum et ne permet en aucun cas de déléguer la responsabilité légale
de l’association.
Tout membre sortant se voit démis de ses fonctions et perd ainsi les droits qui lui étaient attribués.

\subsection{Obligation des membres}
\label{sec:membres:obligation}

\subsubsection{Justificatifs d’adhésion}
\label{sec:membres:obligation:justificatif}

L’étudiant a pour obligation de fournir les documents nécessaires à la constitution et à la mise à jour de son
dossier étudiant. La justification d’adhésion se fera au minimum pour chaque année scolaire du mois de
Septembre jusqu’au mois de Septembre de l’année suivante. Il atteste être étudiant de TELECOM Nancy et
s’engage à faire connaître à \asso~toute modification intervenant dans son cursus universitaire ou
scolaire, ainsi que tout changement de coordonnées et informations pouvant faire évoluer son statut
d’étudiant (adresse, téléphone, adresse e-mail, droits de sécurité sociale étudiante, etc.).

\subsection{Exclusion}
\label{sec:membres:exclusion}

La procédure d’exclusion est proposée par le Président ou par au moins 2 (deux) administrateurs.
Celle-ci est prononcée par le Conseil d’Administration à une majorité renforcée des deux-tiers.
L’exclusion peut être proposée uniquement dans les cas suivants :
\begin{itemize}
	\item Un membre a commis un manquement grave à ses obligations en tant que membre d’\asso. Ce
manquement sera appuyé par la (les) personne(s) qui la propose(nt) par un article de ce Règlement
Intérieur
	\item Un membre n’a pas montré d’intérêt ni d’activité sur aucun canal d’\asso
	\item Communication sur aucun projet porté par \asso~pendant au minimum 6 semaines
	\item Un membre a porté une insulte grave à l’encontre d’un autre membre d’\asso, ou à l’image
d’\asso
	\item Un membre n’a pas fourni les justificatifs obligatoires à son adhésion ou ne les a pas mis à jour malgré
plusieurs relances
\end{itemize}


\subsection{Démission, Décès, Disparition}
\label{sec:membres:demission}

Le membre démissionnaire devra adresser, sous lettre simple signée manuscritement, sa décision au
Président.
En cas de décès, la qualité de membre ainsi que tout autre statut préférentiel s’effacent avec la personne.

\section{Entités décisionnelles}
\label{sec:entites_decisionnelles}

\subsection{Le Conseil d’Administration}
\label{sec:entites_decisionnelles:conseil_admin}

\subsubsection{Rôle}
\label{sec:entites_decisionnelles:conseil_admin:role}

Conformément à l’article 10 (dix) des Statuts de l’association Humani'TN, le Conseil d’Administration a pour
objet de prendre toutes les décisions relatives à tout acte d’administration, de disposition et de gestion.

\subsubsection{Composition}
\label{sec:entites_decisionnelles:conseil_admin:composition}

Il est composé des administrateurs et est présidé par le Président ou un Vice-Président. En cas d’absence de
ces membres, un administrateur sera désigné par convention tacite en début de séance.



Le Conseil d’Administration est constitué des responsables de chaque pôle de l’association ainsi que de
l’ensemble des membres du bureau. Les administrateurs, une personne par poste, occupent au minimum les
postes suivants :
\begin{itemize}
	\item le Président
	\item le Vice-Président
	\item le Trésorier
	\item le Secrétaire Général
	\item le Responsable Communication
	\item le Responsable Logistique
	\item le Responsable Relations Commerciales
	\item le Responsable Association Partenaire
\end{itemize}

Un membre du Conseil d’Administration peut demander à rajouter un poste au Conseil d’Administration.
Cette demande fera l’objet d’un point spécifique à la prochaine réunion du Conseil d’Administration. Un vote
consultatif devra être effectué par les membres du Bureau lors de cette réunion. Cependant, le vote effectif
s’effectuera au sein des membres du Bureau suivant les règles établies dans le point XXX des Statuts.

\subsubsection{Election}
\label{sec:entites_decisionnelles:conseil_admin:election}

Le Conseil d’Administration est élu chaque année conformément à l’article 10 alinéa 3 des Statuts.

\subsubsection{Réunion}
\label{sec:entites_decisionnelles:conseil_admin:reunion}

Lors de la première réunion, le Conseil d’Administration définit son organisation. C’est-à-dire qu’il définit
l’organigramme et son mode de fonctionnement.
Le Conseil d’Administration se réunit sur convocation du Président ou à la demande du tiers de ses membres
au minimum une fois par mois.
Il est possible d’assister au Conseil d’Administration ou d’organiser celui-ci par visioconférence en cas
d’impossibilité d’assurer une présence physique des membres à ce Conseil d’Administration. On privilégiera
toujours une présence physique aux réunions du conseil.
Hors période de congés, le Conseil d’Administration se tient au moins une fois par mois.

\subsubsection{Procuration}
\label{sec:entites_decisionnelles:conseil_admin:procuration}

Dans le cas où un administrateur se retrouve dans l’impossibilité de participer au Conseil d’Administration, il
pourra donner procuration à un autre administrateur.
La procuration est valable pour l’ensemble des points votés lors d’une seule et unique réunion du Conseil
d’Administration. Elle doit obligatoirement expliciter le jour de la réunion et les identités des mandats et
mandataires. La date et la signature du mandant doivent être apposées.
Seul un administrateur est habilité à recevoir une procuration, dans la limite de deux procurations par individu.
La procuration doit être archivée avec la feuille de présence au Conseil d’Administration.

\subsubsection{Prise de décision}
\label{sec:entites_decisionnelles:conseil_admin:decision}

Le quorum est fixé à deux tiers des membres. Dans le cas où le quorum n’est pas atteint, ce dernier est fixé à
la moitié, puis au quart des membres pour les deux Conseils d’Administration suivants.
Les décisions se prennent à la majorité simple des membres présents ou ayant donné procuration à un
membre présent, sauf cas particuliers.
Lors des votes, chaque administrateur ne peut faire valoir qu’une seule voix, cependant la voix du Président
est prépondérante en cas d’égalité.



La procuration doit faire l’objet d’un document écrit et signé ou de l’envoi d’un courrier électronique au
Président de séance.

\subsubsection{Ordre du jour}
\label{sec:entites_decisionnelles:conseil_admin:ordre_jour}

L’ordre du jour est réglé par le Président de séance, en concertation avec le Secrétaire Général. Chaque
membre du Conseil d’Administration a la possibilité de faire apparaître des points à l’ordre du jour.
Lors de sa réunion, le Conseil d’Administration peut exiger de chaque membre du Conseil d’Administration
un compte-rendu de l’état de son Pôle, ainsi que du Trésorier la communication de l’état de la trésorerie.

\subsubsection{Compte-rendu}
\label{sec:entites_decisionnelles:conseil_admin:compte_rendu}

Un compte-rendu de réunion est rédigé par le Secrétaire Général et archivé sur le drive dans le dossier
secrétariat.

\subsection{Le bureau}
\label{sec:entites_decisionnelles:bureau}


\subsubsection{Rôle}
\label{sec:entites_decisionnelles:bureau:role}

Conformément à l’article 9 (neuf) des statuts d’Humani'TN, le bureau a pour objet d’établir la stratégie du
mandat et applique la politique définie par le Conseil d’Administration.
Le Président est responsable de la gestion morale d’Humani'TN. Il est le seul habilité à signer les documents
engageant Humani'TN. Il ordonne les dépenses avec l’accord nécessaire du Trésorier.
Le Président est habilité à effectuer les ordres de paiement.
Le Président peut donner délégation de ses pouvoirs à tout administrateur à l’exception du Trésorier.
Le Trésorier est responsable de la gestion comptable et financière d’Humani'TN. Il assure le recouvrement
des cotisations et des ressources de toute nature.
Dans le cas où au moins un administrateur distinct du Trésorier est affecté à la saisie comptable, le Trésorier
est également habilité à effectuer les ordres de paiement.
Le Trésorier peut donner délégation de ses pouvoirs à tout administrateur à l’exception du Président et de
toute personne affectée à la saisie comptable.

\subsubsection{Réunion}
\label{sec:entites_decisionnelles:bureau:reunion}

Le Bureau se réunit à la demande du Président, ou à la demande de deux (2) de ses membres.
La tenue d’une réunion hebdomadaire ou toutes les deux semaines est fortement recommandée.
Les décisions sont prises à la majorité simple en comptabilisant les votes blancs, la voix du Président étant
prépondérante en cas d’égalité.

\subsubsection{Élection}
\label{sec:entites_decisionnelles:bureau:election}

Le Bureau est élu chaque année conformément à l’article 9 alinéa 5 des Statuts.

\subsubsection{Vacance}
\label{sec:entites_decisionnelles:bureau:vacance}

En cas de vacance d’un poste du Bureau, le Conseil d’Administration procède au remplacement lors de sa
réunion suivante.
Les seuls candidats possibles sont les membres du Conseil d’Administration.
Le vote se fait à bulletin secret ou à main levée après consultation de l’assemblée par le président de séance,
et à la majorité simple en comptabilisant les votes blancs.




\subsection{Assemblée Générale Ordinaire}
\label{sec:entites_decisionnelles:ago}

\subsubsection{Période}
\label{sec:entites_decisionnelles:ago:periode}

Conformément à l’article 11 (onze) des Statuts de l’association, l’Assemblée Générale Ordinaire se réunit une
fois par an au mois de janvier sur convocation du Président.

\subsubsection{Participants}
\label{sec:entites_decisionnelles:ago:participants}

Seuls les membres adhérents de l’association sont autorisés à participer, les membres d’honneurs, les
membres bienfaiteurs ou des personnes extérieures à Humani'TN peuvent être invités à assister aux
délibérations ainsi qu’à prendre part aux discussions mais ils n’ont pas le droit de vote.

\subsubsection{Convocation}
\label{sec:entites_decisionnelles:ago:convocation}

Les participants sont convoqués suivant la procédure suivante :
L’ordre du jour et la convocation sont rédigés par le Secrétaire Général au minimum 2 (deux) semaines avant
le déroulement de l’Assemblée Générale Ordinaire.

\subsubsection{Réunion}
\label{sec:entites_decisionnelles:ago:reunion}

Un président et un secrétaire de séance sont désignés en début d’Assemblée. En cas de non-désignation de
ceux-ci, le Président et le Secrétaire Général de l’association sont nommés tacitement à ces postes respectifs.
Les décisions se prennent à la majorité simple en comptabilisant les votes blancs. En cas d’égalité, la voix du
Président est prépondérante et sera annoncée à haute voix.
L’Assemblée Générale Ordinaire a notamment pour but d’élire le Conseil d’Administration et de valider les
quitus moral et financier de l’exercice écoulé.

\subsection{Assemblée Générale Extraordinaire}
\label{sec:entites_decisionnelles:age}

\subsubsection{Période}
\label{sec:entites_decisionnelles:age:periode}

Conformément à l’article 12 (douze) des Statuts de l’association Humani'TN, une Assemblée Générale
Extraordinaire peut être convoquée en cas de modification des Statuts d’Humani'TN, ainsi que de dissolution
d’Humani'TN. Elle peut également statuer sur la procédure de révocation d’un des membres du bureau.

\subsubsection{Participants}
\label{sec:entites_decisionnelles:age:participants}

Seuls les membres adhérents Humani'TN sont autorisés à participer, les membres d’honneurs et les
personnes extérieures à Humani'TN peuvent être invités à assister aux délibérations ainsi qu’à prendre part
aux discussions mais ils n’ont pas le droit de vote.

\subsubsection{Convocation}
\label{sec:entites_decisionnelles:age:convocation}

Les participants sont convoqués suivant la procédure suivante : L’Assemblée Générale Extraordinaire se
réunit sur convocation du Président ou de 2 (deux) des membres du bureau ou de 50\% du Conseil
d’Administration.

\subsubsection{Réunion}
\label{sec:entites_decisionnelles:age:reunion}

Son ordre du jour est réglé par le Président et le Secrétaire Général au moins deux semaines à l’avance.
Le quorum est fixé au deux tiers des votants. Dans le cas où le quorum n’est pas atteint, l’Assemblée Générale
Extraordinaire est nulle et sans effet.



Le vote se déroule selon les modalités suivantes :
Les votes se font à bulletin secret ou à main levée après consultation de l’assemblée par le président de
séance. Les décisions sont votées à la majorité des deux-tiers (2/3) des présents, la voix du Président étant
prépondérante en cas d’égalité et sera annoncée à haute voix.
Un Procès-Verbal d’Assemblée Générale est rédigé par le Secrétaire Général et archivé dans le cahier
d’Association. Il est signé et paraphé par le Président et le Secrétaire Général.

\subsubsection{Substitution de l’Assemblée Générale Ordinaire}
\label{sec:entites_decisionnelles:age:substitution}

L’Assemblée Générale Extraordinaire peut aussi remplacer une Assemblée Générale Ordinaire si une
modification des Statuts de Humani'TN est décidée au moment de la tenue d’une Assemblée Générale
Ordinaire. Dans ce cas, les attributions de l’Assemblée Générale Ordinaire, telles que définies dans l’alinéa 2
(deux) de l’article 11 (onze) des Statuts, pourront être votées lors de cette Assemblée Générale Extraordinaire
au même titre que la modification des Statuts. Dans ce cas, les décisions concernant les attributions de
l’Assemblée Générale Ordinaire se prennent à la majorité simple en comptabilisant les votes blancs, tel que
défini dans l’alinéa 3 (trois) de l’article 11 (onze) des Statuts.

\section{Dépenses}
\label{sec:depenses}


\subsection{Frais de fonctionnement}
\label{sec:depenses:fonctionnement}

Toute dépense au sein de l’association donne lieu à une autorisation préalable.
Les divers frais de fonctionnement inférieurs à soixante-quinze (75) euros feront l’objet de bons de commande
approuvés par le Trésorier. Les frais d’un montant supérieur à soixante-quinze (75) euros feront l’objet de bons
de commande approuvés par la majorité des membres du Conseil d’Administration.
Tout règlement devra être justifié par les factures correspondantes. Le Trésorier effectue le règlement au vu
des factures, si ces dépenses ont donné lieu à un engagement préalable.
Toute facture comporte :
\begin{itemize}
	\item L’identité du facturant
	\item L’adresse du facturant
	\item L’objet de la facture
	\item La date de la facturation
	\item Le numéro SIRET/SIREN du facturant
\end{itemize}


\subsection{Frais de déplacement}
\label{sec:depenses:deplacement}

Humani'TN remboursera les frais de déplacement de ses membres pour les déplacements concernant :
\begin{itemize}
	\item Déplacement à des fins de prospection
	\item Autres raisons faisant l’objet d’un ordre de mission approuvé par le bureau
\end{itemize}

Pour les déplacements en véhicule personnel, le tarif kilométrique est fixé par le Conseil d’Administration sur
proposition du Trésorier. Le vote se fait à la majorité simple.
Le tarif kilométrique est actuellement fixé à 0,11 €/km.
Le remboursement fait l’objet d’une note de frais signée par le Trésorier, et le membre remboursé.

\section{Dispositions diverses}
\label{sec:dispositions_diverses}


\subsection{Modification du Règlement Intérieur}
\label{sec:dispositions_diverses:modification}


Le Règlement Intérieur d’\asso~est établi par le Conseil d’Administration, conformément à l’article 13
(treize) des Statuts.
Il peut être modifié par le Conseil d’Administration, sur proposition du Président, ou de deux des membres du
Bureau, ou de deux (2) des membres du Conseil d’Administration selon la procédure suivante :
Les personnes proposant la modification du Règlement Intérieur devront fournir une version au Conseil
d’Administration au minimum quatre (4) jours avant la tenue du vote.
Le nouveau Règlement Intérieur sera adressé à chacun des membres de l’association consultable sur le drive
dans le dossier secrétariat et dans le cahier d’Association sous un délai de sept (7) jours suivant la date de la
modification.

\subsection{Tenu des comptes}
\label{sec:dispositions_diverses:tenu_comptes}

Le Trésorier et le Vice-Trésorier établissent pour l’Assemblée Générale Ordinaire le bilan et le compte
d’exploitation de l’association.
Le Trésorier assure quotidiennement, en coopération avec le Vice-Trésorier, la tenue du livre des comptes qui
mémorise tous les mouvements d’argent dus à l’exercice des activités de l’association.
Le livre des comptes, le bilan et le compte d’exploitation sont en permanence à la disposition des membres
de l’association.


\subsection{Archivage}
\label{sec:dispositions_diverses:archivage}


Par archivage, il est à comprendre à la fois un archivage papier, et un archivage numérique pour chaque
document. Chaque type de documents doit être gardé durant une durée bien définie. Voici une liste des
durées d’archivage avec les types de documents correspondants.
Documents devant être gardés 5 ans :

\begin{itemize}
    \item Factures
    \item Fiches d’indemnisation
    \item Bons de commandes
    \item Rapport global
    \item Comptes annuels
    \item Liasse fiscal
    \item Journaux
    \item Grand livre
    \item Balance globale
    \item Compte rendu des Conseils d’Administration
\end{itemize}


Documents devant être gardés 10 ans :

\begin{itemize}
    \item Contrats d’assurance
    \item Tous les documents liés aux déclaratifs fiscaux et sociaux (URSSAF, TVA, IS, DADS, DAD2)
    \item Déclarations CNIL
    \item Lettres de délégation de pouvoirs
    \item Dossiers d’adhésion
    \item Preuves comptables (facture, quittance, chèque, effet de commerce ...)
\end{itemize}



Documents devant être archivés jusqu’à la dissolution de l’association :

\begin{itemize}
    \item Statuts
    \item Règlements Intérieurs
    \item Extrait du Journal Officiel
    \item Récépissés de Préfecture
    \item Procès-verbal d’Assemblée Générale
\end{itemize}


\end{document}